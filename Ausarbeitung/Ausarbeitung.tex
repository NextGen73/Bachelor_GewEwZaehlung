\documentclass[ngerman,BCOR=4mm]{tudscrreprt}

\usepackage[T1]{fontenc}
\usepackage[ngerman=ngerman]{hyphsubst}
\usepackage[ngerman]{babel}
\usepackage{isodate}
\usepackage[hidelinks]{hyperref}
\usepackage{amsmath, amsfonts, amssymb, amsthm} %% mathematics tools
\usepackage{csquotes}
\usepackage[draft]{listofsymbols}
\usepackage[backend=biber, style=numeric]{biblatex} %% Literature citing engine

%__________Definitions_____________________________

% Literaturverzeichnis
\addbibresource{Literatur.bib}


% Symbolverzeichnis
\renewcommand{\symheadingname}{Symbolverzeichnis}
\opensymdef
      \newsym[afd]{al}{\alpha}
\closesymdef


% wird für Symbolverzeichnis benutzt, sonst gibt es Fehler
\DeclareOldFontCommand{\bf}{\normalfont\bfseries}{\mathbf}


% Math
\theoremstyle{plain} % text is cursive
\newtheorem{theorem}{Theorem}
\newtheorem{lemma}[theorem]{Lemma}  %% [theorem] means same numbering for theorem and lemma
\newtheorem{proposition}[theorem]{Proposition}
\newtheorem{corollary}[theorem]{Korollar}

\theoremstyle{definition} % text is "upright"
\newtheorem{definition}[theorem]{Definition}
\newtheorem{example}[theorem]{Beispiel}

\theoremstyle{remark}
\newtheorem{remark}[theorem]{Bermerkung}

\begin{document}
\selectlanguage{ngerman}

%__________Deckblatt_Allg_Infos_____________________________
\faculty{Fakultät Mathematik}
\institute{Institut für Numerische Mathematik}
% \chair{Professur für Numerik partieller Differentialgleichungen}
\date{23.08.2024}
\title{%
      Gewichtete Eigenwert-Zählung auf einem Intervall
}
\subject{bachelor}
\graduation[B.Sc.]{Bachelor of Science}
\author{%
      Noah Göpel%
      \matriculationnumber{5029415}%
      \dateofbirth{01.02.2003}%
      \placeofbirth{Riesa}%
}
\matriculationyear{2021}
% \supervisor{}
\professor{Prof. Dr. Oliver Sander}
\maketitle

\clearpage
\tableofcontents
\clearpage
\listofsymbols
\clearpage

\chapter{Einleitung}
\label{sec: Einleitung}
      In dieser Ausarbeitung werden die gewichtete Zählung von Eigenwerten auf einem Intervall behandelt.
      Dazu wird in den ersten Kapiteln die nötige Theorie anhand der Quellen \cite{grundlageFutamura} und \cite{hauptteilTkachuk} beschrieben und anschließend in Kapitel \ref{sec: Programmieren} in Python angewendet.
      Daraufhin werden in Kapitel \ref{sec: Verbesserungen} die Ergebnisse kritisch betrachtet und weitere Algorithmen eingeführt, um die Berechnungen zu beschleunigen.
      Zum Schluss werden in Kapitel \ref{sec: Auswertung} die Ergebnisse wiederholt und ein Ausblick in weiterführende Themen gegeben.

\chapter{Masse- und Steifigkeits-Matrizen}
\label{sec: MS Matrizen}

      

\chapter{Das verallgemeinerte Eigenwert-Problem}
\label{sec: EW Problem}

\chapter{Quellen-Ausarbeitungen}

\chapter{Implementierung}
\label{sec: Programmieren}

\chapter{Verbesserungen}
\label{sec: Verbesserungen}

\chapter{Auswertung}
\label{sec: Auswertung}

\chapter{Literaturverzeichnis}
\printbibliography
\end{document}
