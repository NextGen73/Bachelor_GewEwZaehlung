% !TeX spellcheck = de_DE
\documentclass[a4paper,12pt]{report}
% es wurde leqno entfernt, da damit die Nummern von Gleichungen links standen
\usepackage{inputenc,fontenc}
\usepackage[a4paper,margin=3cm]{geometry}
\usepackage[english, german]{babel}
\usepackage[ngerman=ngerman]{hyphsubst}
% \usepackage{isodate}
\usepackage[hidelinks]{hyperref}
\usepackage{amsmath, amsfonts, amssymb, amsthm} %% mathematics tools
\usepackage{csquotes}
\usepackage[draft]{listofsymbols}
\usepackage[backend=biber, urldateusetime=true, sorting=none]{biblatex} %% Literature citing engine
\usepackage{subcaption}
\usepackage{graphicx}
\usepackage{dsfont} % für charakteristische 1

%%
%% Pflichtangaben -bitte hier einsetzen %%%%%%%%%%%%%%%%%%%%%%%%%%%%%%%%%%%%%%%%%%%%%%
%%
\newcommand{\name}{Göpel}
\newcommand{\vorname}{Noah}
\newcommand{\gebdatum}{01.02.2003}
\newcommand{\ort}{Riesa}
\newcommand{\betreuer}{Prof. Dr. Oliver Sander}
\newcommand{\institut}{Institut für Numerische Mathematik}
\newcommand{\thema}{Gewichtete Eigenwert-Zählung auf einem Intervall}
\newcommand{\datum}{tt.\ mm.\ jjjj} %Format tt.\ mm.\ jjjj

%%%%%%%%%%%%%%%%%%%%%%%%%%%%%%%%%%%%%%%%%%%%%%%%%%%%%%%%%%%%%%%%%%%%%%%%%%%%%%%%%%%%%%


%__________Definitions_____________________________

% Literaturverzeichnis
\addbibresource{Literatur.bib}
\graphicspath{{src/}}

% \newcommand{\bild}[2]{\includegraphics[width=#2\textwidth,height=#2\textheight,keepaspectratio]{#1}}
\newcommand{\bild}[4]{
      \begin{figure}[!htp]
            \centering
            \includegraphics[width=#2\textwidth,height=#2\textheight,keepaspectratio]{#1}
            \caption{#3}
            #4
      \end{figure}
}
\newcommand{\R}{\mathbb R}
\newcommand{\C}{\mathbb C}
\newcommand{\zitat}[1]{\glqq #1\grqq}
\newcommand{\klammer}[1]{\left(#1\right)}
\newcommand{\diag}{\text{diag}}
\newcommand{\rang}{\text{rang}}
\newcommand{\tr}{\text{tr}}
\newcommand{\AlamB}{A-\lambda\,B}
\newcommand{\Cnn}{\C^{n\times n}}
\newcommand{\inv}{^{-1}}
\newcommand{\1}{\mathds{1}}
\newcommand{\Res}{\text{Res}}
\newcommand{\interior}{\text{int}}

% wird für Symbolverzeichnis benutzt, sonst gibt es Fehler
\DeclareOldFontCommand{\bf}{\normalfont\bfseries}{\mathbf}

% Math
\theoremstyle{plain} % text is cursive
\newtheorem{theorem}{Theorem}
\newtheorem{lemma}[theorem]{Lemma}  %% [theorem] means same numbering for theorem and lemma
\newtheorem{proposition}[theorem]{Proposition}
\newtheorem{corollary}[theorem]{Korollar}

\theoremstyle{definition} % text is "upright"
\newtheorem{definition}[theorem]{Definition}
\newtheorem{example}[theorem]{Beispiel}

\theoremstyle{remark}
\newtheorem{remark}[theorem]{Bermerkung}

% Symbolverzeichnis
\renewcommand{\symheadingname}{Symbolverzeichnis}
\opensymdef
      % \newsym[]{}{}
      \newsym[Einheitsmatrix]{In}{I_n}
      \newsym[Massenmatrix]{M}{M}
      \newsym[Steifigkeitsmatrix]{K}{K}
      \newsym[Federkraft]{FL}{\overrightarrow{F_L}}
      \newsym[Trägheitskraft]{FT}{\overrightarrow{F_T}}
      \newsym[Auslenkung des k-ten Massepunktes aus der Ruhelage]{xk}{x_k}
      \newsym[Eigenkreisfrequenz]{w}{\omega}
      \newsym[vorgegebenes Intervall]{wAwB}{[\omega_a,\omega_b]}
      \newsym[Eigenwert des Matrix-Pencils, definiert durch $\lambda:=\omega^2$]{lam}{\lambda}
      \newsym[Intervall, definiert durch $\lbrack\lambda_a,\lambda_b\rbrack := \lbrack\omega_a,\omega_b\rbrack$]{lamAlamB}{[\lambda_a,\lambda_b]}
\closesymdef
% , soll zum Schluss frei von Eigenkreisfrequenzen sein

\begin{document}
\selectlanguage{german}

%% Titelseite
\thispagestyle{empty}

\begin{center}
{\Large Technische Universit\"{a}t Dresden\  \ \textbullet\ \ Fakult\"{a}t Mathematik}

\vfil

{\bfseries\Huge\thema}

\vfil
{\LARGE
Bachelorarbeit \\[\bigskipamount]
zur Erlangung des ersten Hochschulgrades\\[\bigskipamount]
\bfseries{\itshape Bachelor of Science  \textup{(}B.Sc.\textup{)}}\\[\bigskipamount]
}

\vfil\vfil

\vfil

vorgelegt von
\\[\bigskipamount]
\textsc{\vorname\ } \MakeUppercase{\name}
\\[\bigskipamount]
(geboren am \gebdatum\ in \textsc{\ort})
\\[\bigskipamount]
Tag der Einreichung: \datum
\\[\bigskipamount]
\betreuer\ (\institut)
\end{center}

\cleardoublepage
%%%%%%%%%%%%%%%%%%%%%%Beginn Ausarbeitung%%%%%%%%%%%%%%%%%%%%%%%%%%%%%%%%
\tableofcontents
\clearpage
\listofsymbols
\clearpage

\chapter{Einleitung}
\label{sec: Einleitung}
      
      Diese Ausarbeitung beschäftigt sich mit numerischen Verfahren, um sicherzustellen, dass für ein gegebenes mechanisches System keine Eigenwerte in einem vorgegebenem festen Intervall liegen.
% hier muss noch einiges getan werden
      Damit kann sichergestellt werden, dass die Eigenfrequenzen des Systems nicht so liegen, dass es zu einer Selbsterregung kommt.

      % Um diese Anforderung zu erfüllen, werden die Eigenwerte des Matrix-Pencils auf diesem Intervall gewichtet gezählt. Man erhält ein Minimierungsproblem,
      % in welchem es gilt, einen Design-Paramerter so anzupassen, dass auf dem Intervall kein Eigenwert des entsprechenden Systems mehr vorhanden ist.
      % Dazu werden in den Kapiteln \ref{sec: MS Matrizen} und \ref{sec: EW Problem} die theoretischen Grundlagen gelegt.
      % Anschließend werden in Kapitel \ref{sec: Quellen} die entscheidende Identität von Futamura und wichtige Überlegungen ausgeführt.

      % Daraufhin werden diese Überlegungen in Kapitel \ref{sec: Programmieren} anhand verschiedener Beispiele implementiert und getestet.

      Die Auswertung und Verbesserung der Implementation folgt in Kapitel \ref{sec: Verbesserungen}.

      % In dieser Ausarbeitung werden die gewichtete Zählung von Eigenwerten auf einem Intervall behandelt, um sicherzustellen,
      % dass bei einem vorgegebenem mechanischem System keine Eigenwerte in einem bestimmten festen Intervall liegen.
      % Dazu wird in den ersten Kapiteln die nötige Theorie anhand der Quellen \cite{grundlageFutamura} und \cite{hauptteilTkachuk} beschrieben.
      % Hier werden Matrix-Pencil und das allgemeinerte Eigenwertproblem verwendet, um 
      % Hierzu nutzt man die Identität von Futamura, um die Zählung der Eigenwerte auf die zugrundeliegenden Matrizen zurückzuführen.


      % und anschließend in Kapitel \ref{sec: Programmieren} in Python angewendet.
      % Daraufhin werden in Kapitel \ref{sec: Verbesserungen} die Ergebnisse kritisch betrachtet und weitere Algorithmen eingeführt, um die Berechnungen zu beschleunigen.
      Zum Schluss werden in Kapitel \ref{sec: Auswertung} die Ergebnisse wiederholt und ein Ausblick in weiterführende Themen gegeben.

      Ferner werden in Kapitel \ref{sec: Verbesserungen} eine Approximation der Matrix-Spur vorgestellt und ebenfalls implementiert.

\chapter{Das allgemeine Eigenwertproblem und die Identität von Futamura}
\label{sec: EW Problem_Futamura}

      Zu Beginn der Ausarbeitung werden einige Resultate über Matrizen und lineare Algebra behandelt.
      Diese werden in Kapitel \ref{sec: EW Zählung} verwendet, um die Zählung der Eigenwerte zu einer differenzierbaren Funktion umzuformen.
      
      \section{Der Matrix-Pencil}
            Betrachte zunächst das Konzept des \zitat{Matrix-Pencils} \cite[S. 32]{matrixPencilDeutsch}, welches in dieser Ausarbeitung eine entscheidende Rolle spielt.
            \begin{definition}(pencil, \cite[S. 375]{matrixGolub})
                  \label{def: pencil}
                  Seien $A$ und $B$ Matrizen in $\C^{n\times n}$, dann wird die Menge aller Matrizen der Form
                  $\AlamB, \lambda \in \C$ Matrix-Pencil genannt.
            \end{definition}

            Laut \cite[S. 375]{matrixGolub} seien die Eigenwerte von $\AlamB$ diejenigen $\lambda \in\C$, für die gelte:
            \begin{equation}
                  \label{eqn: allg EW Problem}
                  (\AlamB)x=0,\quad x\ne 0
            \end{equation}
            
            Hier wird $x\in\C^n$ Eigenvektor genannt.
            Man beachte, dass für $B=\In$ Gleichung (\ref{eqn: allg EW Problem}) die folgende Form annimmt:
            $$(A-\lambda\In)x=0 \Leftrightarrow Ax = \lambda x,$$
            also genau die Form des speziellen Eigenwertproblems.
% gibt es ein Eigenwertproblem und wenn ja, gibt es eine Definition? oder ist es einfach eine Gleichung?
            Daher wird die Suche nach den $\lambda\in\C$, die (\ref{eqn: allg EW Problem}) erfüllen, auch \zitat{allgemeines Eigenwertproblem}\cite[S. 380]{maschinendynamikDresig} genannt, da sie eine allgemeine Form des speziellen Eigenwertproblems darstellen.
            Falls man $\AlamB$ durch die Matrix $C$ ersetzt, so entsteht aus (\ref{eqn: allg EW Problem}) das lineare homogene Gleichungssystem
            $$Cx=0,\quad x\ne 0$$

            Nach einem Resultat aus der linearen Algebra gilt:
% welches Resultat?, ich habe es bis jetzt nur in Skript LA20 gefunden
            $$\det(C)\ne 0 \Leftrightarrow Cx=0 \text{ hat als einzige Lösung }x=0$$
            Durch Negation folgt:
            $$\det(C)=0 \Leftrightarrow \exists x\ne 0: Cx=0$$

            Somit sucht man die Eigenwerte des Matrix-Pencils, in dem man die Nullstellen des Polynoms $\det(\AlamB)$ berechnet\footnote{Man könnte das Polynom $\det(\AlamB)$ als eine Art charakteristisches Polynom des Matrix-Pencils $\AlamB$ betrachten, ähnlich dem charakteristischen Polynom $f(A)$ einer Matrix $A$}.
            
            Daher wird in \cite[S. 375]{matrixGolub} die Menge der Eigenwerte des Matrix-Pencils $\AlamB$ wie folgt definiert:
            \begin{equation}
                  \label{def: EW Pencil}
                  \lambda(A,B):=\{z\in\C:\ \det(A - zB) = 0\}
            \end{equation}

            Laut \cite[S. 375]{matrixGolub} gelte zudem:
            \begin{equation}
                  \label{eqn: n EW äquiv rang B n}
                  \AlamB \text{ hat }n\text{ Eigenwerte}\Leftrightarrow \rang(B)=n
            \end{equation}
            Dieses Resultat wird für Kapitel \ref{sec: EW Zählung} relevant werden und kann mithilfe von Theorem \ref{thrm: allg Schur Zerlegung} gezeigt werden.

            Man benötigt für die Identität von Futamura aus Kapitel \ref{sec: Futamura} zudem eine weitere Definition:
            \begin{definition}(Regulärer Matrix-Pencil, vgl. \cite[S. 784]{regularMatrixPencil})\\
                  \label{def: regulärer Pencil}
                  Ein Matrix-Pencil $A+\lambda B$ wird regulär genannt, wenn $A,B\in \Cnn$ und
                  $$\det(A+\lambda B)\ne 0$$ 
            \end{definition}

            Beachte, dass hier zwar der Matrix-Pencil mit $+$ statt $-$ angegeben wurde, aber diese Definitionen sind gleichwertig, da man $\tilde \lambda := -\lambda$ in den Matrix-Pencil einsetzen kann.
            \begin{remark}
                  \label{bem: B reg impl pencil reg}
                  Diese Definition bedeutet insbesondere, dass der Matrix-Pencil regulär ist, falls B regulär ist, denn es gilt:
                  \begin{align*}
                        B\text{ regulär} \Leftrightarrow & \rang(B)=n\Leftrightarrow |\lambda(A,B)| = n<\infty\\
                        \Rightarrow & \exists z\in\C\setminus\lambda(A,B): \det(A+zB) \ne 0 \Leftrightarrow \AlamB \text{ regulär}
                  \end{align*}
                  Wobei (\ref{eqn: n EW äquiv rang B n}) und die Überabzählbarkeit von $\C$ verwendet wurde.
            \end{remark}
            
% man kann auch noch Korollar 2.3.11 (LinAlg Werner, S. 50) ansprechen
      \section{Die Schur-Zerlegung}
            In diesem Kapitel werden weitere wichtige Konzepte vorgestellt, die für Kapitel \ref{sec: Futamura} benötigt werden.

            \begin{definition}(\cite[S. 73]{matrixGolub})
                  Sei $Q \in\C^{n\times n}$, dann wird $Q$ unitär genannt, wenn
                  $$Q^HQ = QQ^H=\In$$
            \end{definition}

            Hier ist $Q^H$ die Adjungierte von $Q$, also die Matrix, die durch komplexe Konjugation und Transponierung von $Q$ entsteht, siehe auch \cite[S. 14]{matrixGolub}.

%             \begin{theorem}(vgl. Theorem 7.1.3 (Schur Decomposition)\cite[S. 313]{matrixGolub})\\
%                   \label{thrm: Schur Zerlegung}
%                   Wenn $A \in \C^{n\times n}$, dann existiert eine unitäre Matrix $Q\in\C^{n\times n}$, sodass
%                   \begin{equation}
%                         \label{eqn: Schur_Resultat}
%                         Q^H\,AQ = T = D+N
%                   \end{equation}
%                   mit $D=\diag(\lambda_1,\dots,\lambda_n)$ und $N\in\C^{n\times n}$ strikter oberer Dreiecksmatrix.
%                   Ferner kann $Q$ so gewählt werden, dass die Eigenwerte $\lambda_i$ in beliebiger Reihenfolge auf der Diagonalen auftauchen.
%             \end{theorem}

%             Um dieses Theorem zu beweisen, benötigt man ein Hilfslemma aus \cite[S.312]{matrixGolub}.
%             Das folgende Resultat wird einfachheitshalber in dem speziellen Fall vorgestellt, wie es für den Beweis von Theorem \ref{thrm: Schur Zerlegung} benötigt wird.

%             \begin{lemma}(vgl. Lemma 7.1.2\cite[S. 312]{matrixGolub})
%                   \label{lemma: Hilfe Schur}\\
%                   Falls ein $A\in\C^{n\times n}, \lambda\in \C$ Eigenwert von $A$ und $X\in \C^{n}$
%                   \begin{equation}
%                         \label{eqn: Hilfslemma Schur_Bed}
%                         AX = X\lambda,\qquad \rang(X)=1
%                   \end{equation}
                  
%                   erfüllen, dann existiert ein $Q\in \C^{n\times n}$ unitär, sodass
%                   $$Q^H\,AQ = T =  \begin{bmatrix}
%                         \lambda & T_{12} \\
%                         0 & T_{22} \\
%                         \end{bmatrix}$$
%                   mit $T_{22}\in \C^{(n-1)\times(n-1)}$
%             \end{lemma}

%             Beweis:
%             Erhalte mit \cite[S. 233]{matrixGolub} eine komplexe QR-Zerlegung für $X$. Es gilt dann
%             $$X = Q\,R$$
%             mit $Q\in \C^{n \times n}$ unitär und $R = [r,0,\dots,0]^\top\in \C^n$.
%             Durch Einsetzen in (\ref{eqn: Hilfslemma Schur_Bed}) und Multiplikation mit $Q^H$ folgt
%             \begin{equation}
%                   \label{eqn: Hilfslemma_Schur_Gl}
%                   \underbrace{Q^H\,AQ}_{=:T}\ R = Q^H\,Q\ R\ \lambda
%             \end{equation}
% % man könnte hier R statt dem Vektor nutzen, dann wäre es kompakter
%             hier ist noch $T:=\begin{matrix}
%                   \begin{bmatrix}
%                         T_{11} & T_{12}\\
%                         T_{21} & T_{22}\\
%                   \end{bmatrix} & \begin{matrix}
%                   \text{\small 1} \\\text{\small{n-1}}
%                   \end{matrix} \\
%                   \begin{matrix}
%                   \text{\small 1} &  \text{  \small{n-1}}\\
%                   \end{matrix} &  \\
%                   \end{matrix}$\\

%             Durch Ausmultiplizieren von (\ref{eqn: Hilfslemma_Schur_Gl}) folgen die Gleichungen\\

%             $$\begin{aligned}
%                   T_{11}\,r =& r\,\lambda\\
%                   T_{21}\,r =& 0
%             \end{aligned}$$

%             und mit $\rang(X)=1 \Rightarrow X\ne 0 \Rightarrow r\ne 0$ folgt $T_{11} = \lambda,\ T_{21} = 0$.\qed

%             Nach \cite[S. 312]{matrixGolub} gelte ferner $\lambda(A) = \lambda(T) = \{\lambda\}\cup\lambda(T_{22})$.\\
            

%             Nun können wir Theorem \ref{thrm: Schur Zerlegung} beweisen.

%             Beweis Theorem \ref{thrm: Schur Zerlegung} (vgl. \cite[S. 313]{matrixGolub}):

%             Offenbar gilt für $n=1$:\\
%             $A$ ist ein Skalar, wodurch (\ref{eqn: Schur_Resultat}) für alle $Q$ unitär erfüllt ist.\\
%             Sei nun $n>1$ und (\ref{eqn: Schur_Resultat}) gelte für alle $k<n$, dann gilt für einen beliebigen Eigenwert $\lambda$ von $A$ und zugehörigen Eigenvektor $X$:
%             $$AX = \lambda X = X \lambda,\quad X\in \C^n, X\ne 0$$
%             Nach Lemma \ref{lemma: Hilfe Schur} existiert $U\in \C^{n\times n}$ unitär, sodass
%             $$U^H\, AU = \begin{bmatrix}
%                   \lambda & T_{12} \\
%                   0 & T_{22} \\
%                   \end{bmatrix}$$
%             Da aber $T_{22}\in\C^{(n-1)\times(n-1)}$, so gilt nach der Induktionsvoraussetzung:
%             $$\widetilde U^H\,T_{22}\widetilde U = R,$$
%             wobei $R\in\C^{(n-1)\times(n-1)}$  rechte obere Dreiecksmatrix, $\widetilde U \in\C^{(n-1)\times(n-1)}$ unitär.

%             Sei nun $Q:=U\cdot\begin{bmatrix}
%                   1&0\\
%                   0&\widetilde U
%             \end{bmatrix}$

%             dann gilt mit\footnote{die folgende Gleichung stammt aus \cite{conjugateTranspose}, Beweis ebenfalls dort zu finden}
% % Fussnote vielleicht scheisse platziert
%             \begin{equation}
%                   \label{eqn: Produkt Adjungierter}
%                   \klammer{AB}^H = B^HA^H,\quad A,B\in \C^{n\times n}
%             \end{equation}
%             für (\ref{eqn: Schur_Resultat}):
%             $$\begin{aligned}
%                   Q^H\,AQ =& \begin{bmatrix}
%                         1&0\\
%                         0&\widetilde U
%                   \end{bmatrix}^H\,U^H\,AU \begin{bmatrix}
%                         1&0\\
%                         0&\widetilde U
%                   \end{bmatrix} = \begin{bmatrix}
%                         1&0\\
%                         0&\widetilde U^H
%                   \end{bmatrix}
%                   \begin{bmatrix}
%                         \lambda & T_{12} \\
%                         0 & T_{22} \\
%                   \end{bmatrix}
%                   \begin{bmatrix}
%                         1&0\\
%                         0&\widetilde U
%                   \end{bmatrix}\\
%                   =& \begin{bmatrix}
%                         \lambda&T_{12}\\
%                         0&\widetilde U^H\, T_{22}
%                   \end{bmatrix}
%                   \begin{bmatrix}
%                         1&0\\
%                         0&\widetilde U
%                   \end{bmatrix}
%                   = \begin{bmatrix}
%                         \lambda&T_{12}\,\widetilde U\\
%                         0&\widetilde U^H\, T_{22}\, \widetilde U
%                   \end{bmatrix}\\
%                   =& \begin{bmatrix}
%                         \lambda&T_{12}\,\widetilde U\\
%                         0&R
%                   \end{bmatrix}
%             \end{aligned}$$
%             Wobei verwendet wurde, dass nach Definition die komplex Konjugation skalar angewendet wird\footnote{siehe \cite[S. 14]{matrixGolub}}, also es gilt:
           
%             \begin{equation}
%             \label{eqn: Adjungierte reinziehen}
%                   \begin{bmatrix}
%                         1&0\\
%                         0&\widetilde U
%                   \end{bmatrix}^H = \overline{\begin{bmatrix}
%                         1&0\\
%                         0&\widetilde U^T
%                   \end{bmatrix}} = \begin{bmatrix}
%                         \overline{1}&0\\
%                         0&\overline{\widetilde U^T}
%                   \end{bmatrix} = \begin{bmatrix}
%                         1&0\\
%                         0&\widetilde U^H
%                   \end{bmatrix}
%             \end{equation}

%             Damit ist gezeigt, dass $Q^H\,AQ$ eine rechte obere Dreiecksmatrix ist.

%             $D = \diag(\lambda_1,\dots,\lambda_n)$ folgt mit Korollar 4.2.3 aus \cite[S. 80]{LinAlgWerner}.
%             Die Beliebigkeit der Anordnung der Eigenwerte auf der Diagonalen erhält man durch rekursives Anwenden von Theorem \ref{thrm: Schur Zerlegung} auf die Matrix $T_{22}$.
%             Es fehlt noch zu zeigen, dass $Q$ unitär ist, was aber mit (\ref{eqn: Produkt Adjungierter}) und (\ref{eqn: Adjungierte reinziehen}) einfach gezeigt werden kann. \qed\\

            Mithilfe dieser Definition kann nun die allgemeine Schur-Zerlegung vorgestellt werden.
            Diese wird für den Beweis der Identität von Futamura in Kapitel \ref{sec: Futamura} benötigt.
%hier kann noch viel mehr erzählt werden: die Eigenschaft der Eigenwerte ist von entscheidender Relevanz
            \begin{theorem}(Generalized Schur Decompositon, \cite[S. 377]{matrixGolub})
                  \label{thrm: allg Schur Zerlegung}
                  Seien $A, B \in \C^{n\times n}$, dann existieren $Q$ und $Z$ unitär, sodass
                  \begin{equation}
                        \label{eqn: allg Schur_Resultat}
                        Q^H\,AZ = T,\quad Q^H\, BZ = S
                  \end{equation}
                  mit $T$ und $S$ obere Dreiecksmatrizen.
                  Falls für ein $k\in \{1,\dots, n\}\ t_{kk}=s_{kk}=0$, dann gilt $\lambda(A, B) = \C$, sonst
                  \begin{equation}
                        \label{eqn: EW Pencil nach Schur}
                        \lambda(A, B) = \{t_{ii}/s_{ii}:\, s_{ii}\ne 0\}
                  \end{equation}
            \end{theorem}
            Beweis: siehe \cite[S. 377]{matrixGolub}\qed
      
      \section{Die Identität von Futamura}
      \label{sec: Futamura}

            Nach den Vorbereitungen, die in vorherigen Abschnitten gemacht wurden, kann nun die Identität von Futamura beschrieben und gezeigt werden.

            Das folgende Theorem wird für das Finden einer Zielfunktion von entscheidender Relevanz sein,
            da durch diese Aussage in Kapitel \ref{sec: EW Zählung} ein Zusammenhang zwischen der Anzahl der Eigenwerte auf einem Intervall und einem Kurvenintegral über Matrizen gezeigt werden kann.

            \begin{theorem}(vgl. \cite[S. 127]{grundlageFutamura})
                  \label{thrm: IdentitätFutamura}\\
                  Seien $A, B\in\Cnn, z\in \C$, sodass $(zB-A)$ ein regulärer Matrix-Pencil ist, dann gibt es unitäre Matrizen $Q$ und $U \in\Cnn$, sodass
                  $$A = QTU^H,\quad B=QSU^H$$
                  und
                  \begin{equation}
                        \label{eqn: Resultat_Futamura}
                        \tr((zB-A)\inv B) = \sum_{j\in N}\frac{1}{z-\lambda_j},
                  \end{equation}
                  gilt, wobei $N:=\{i\in\{1,\dots,n\}: s_{ii}\ne 0\}$ und $\lambda_j$ die Eigenwerte des Matrix-Pencils sind.
                  $S$ und $T$ sind hier obere Dreiecksmatrizen.
            \end{theorem}
% warum zB-A? für Eigenwerte macht es keinen Unterschied
            Beweis:
            Nach Theorem \ref{thrm: allg Schur Zerlegung} gibt es unitäre Matrizen $Q$ und $U\in\Cnn$, sodass nach entsprechender Multiplikation gilt:
            $$A=QTU^H,\quad B=QSU^H$$
            mit $T,S\in\Cnn$ obere Dreiecksmatrizen.
            
            Hierbei sind $S$ und $T$ obere Dreiecksmatrizen, und es gilt:
            $$\lambda(B)=\lambda(S) = \{s_{ii}:i=1,\dots,n\},$$
            aber diese Eigenwerte sind noch ungeordnet und man kann nicht wie in \cite{grundlageFutamura}:
            \begin{align*}
                  s_{ii} = \begin{cases}
                        \ne 0 & : i\le n'\\
                        0 & : i>n'
                  \end{cases}
            \end{align*}
            annehmen für $n':=|\{i\in\{1,\dots,n\}: s_{ii}\ne 0\}|$.

            Mithilfe der Definition
            $$N:=\{i=1,\dots,n: s_{ii}\ne 0\}$$
            kann dies aber umgangen werden.

            Man kann folgern:
            \begin{align*}
                  (zB-A)\inv B =& (z\,QSU^H-QTU^H)\inv QSU^H = (Q(zS-T)U^H)\inv QSU^H \\
                  =& U(zS-T)\inv Q^H\,QSU^H = U(zS-T)\inv S\,U\inv
            \end{align*}

            Man sieht, dass $(zB-A)\inv B$ und  $(zS-T)\inv S$ ähnlich zueinander sind.\\
            Da die Spur invariant unter Ähnlichkeitstransformation ist\footnote{vgl. 3. Exercise unter Example 1.3.5 in \cite{matrixSpur}}, gilt
            \begin{equation}
                  \label{eqn: Haltepunkt Bew Futamura}
                  \tr((zB-A)\inv B) = \tr((zS-T)\inv S) = \sum_{i=1}^{n}((zS-T)\inv S)_{ii}
            \end{equation}

            Für die folgenden Beweisschritte benötigt man einige Aussagen über obere Dreiecksmatrizen, hierbei ist mit $X_{ij}$ der Eintag von X in Zeile $i$ und Spalte $j$ gemeint.

            \begin{lemma}
                  \label{Hilfslemma_Futamura: Inv Dreieck}
                  Die Inverse einer oberen Dreiecksmatrix $A$ ist eine obere Dreiecksmatrix
            \end{lemma}
            Beweis: Anwenden des Gauß-Jordan-Algorithmus zur Bestimmung der Inversen auf $(A|\In)$ mit $A$ rechte obere Dreiecksmatrix\qed
% man könnte auch mit Gramerschen Regel argumentieren, aber da ist es schwer zu beweisen, dass die Kofaktoren null sind
% hier gibt es noch Verbesserungsbedarf

            \begin{lemma}
                  \label{Hilfslemma_Futamura: Prod Dreieck}
                  Seien X und Y obere Dreiecksmatrizen, dann gilt:
                  $$(XY)_{ii} = X_{ii}\, Y_{ii}$$
            \end{lemma}
            Beweis:
            $$(XY)_{ii} = \sum_{k=1}^n X_{ik}Y_{ki} = \sum_{k=I}^{n}X_{ik}Y_{ki} = \sum_{k=I}^{i}X_{ik}Y_{ki} = X_{ii} Y_{ii}$$
            wobei verwendet wurde, dass für eine obere Dreiecksmatrix $A$ nach Definition gilt:
            $A_{ij} = 0 \text{ für }i>j$\qed

            \begin{lemma}
                  \label{Hilfslemma_Futamura: Diag Inv Dreieck}
                  Für eine invertierbare obere Dreiecksmatrix X gilt:
                  $$(X\inv)_{ii} = \frac 1 {(X)_{ii}}$$
            \end{lemma}
            Beweis: Sei $i\in\{1,\dots,n\}$ fest, dann gilt
            $$\In = X\,X\inv \Rightarrow 1 = X_{ii}(X\inv)_{ii} \Rightarrow (X\inv)_{ii} = \frac 1 {(X)_{ii}}$$
            Hier wurde Lemma \ref{Hilfslemma_Futamura: Prod Dreieck} und
            $$X \text{ invertierbar}\Rightarrow \det X = \prod_{k=1}^{n}X_{kk}\ne 0\Rightarrow X_{kk}\ne 0\quad \forall k=1,\dots,n$$
            verwendet.\qed\\

            Nun kann man den Beweis für Theorem \ref{thrm: IdentitätFutamura} vervollständigen:
            Mit den Lemmata \ref{Hilfslemma_Futamura: Inv Dreieck}, \ref{Hilfslemma_Futamura: Prod Dreieck} und \ref{Hilfslemma_Futamura: Diag Inv Dreieck} folgt für (\ref{eqn: Haltepunkt Bew Futamura}):
            \begin{align*}
                  \tr((zB-A)\inv B) =& \sum_{i=1}^{n}((zS-T)\inv)_{ii}\ S_{ii} = \sum_{i=1}^{n}((zS-T)_{ii})\inv\ S_{ii}\\
                  =& \sum_{i=1}^{n}\frac{s_{ii}}{z\,s_{ii}-t_{ii}} = \sum_{i\in N}\frac{s_{ii}}{s_{ii}(z-t_{ii}\, s_{ii}\inv)}\\
                  =& \sum_{i\in N} \frac{1}{z-\lambda_i}
            \end{align*}

            mit
            $$\lambda_i = \frac{t_{ii}}{s_{ii}},\quad i\in N.$$

            Nach Theorem \ref{thrm: allg Schur Zerlegung} und der Definition von $N$ gilt:

            $$\lambda(A, B) = \left\{\frac{t_{ii}}{s_{ii}} : s_{ii}\ne 0\right\} = \{\lambda_i: i\in N\}$$
            somit sind die Eigenwerte des Matrix-Pencils genau die Polstellen der Funktion $\tr((zB-A)\inv B)$.

            In Kapitel \ref{sec: EW Zählung} wird daher das Residuentheorem auf ein Kurvenintegral\\
            über $\tr((zB-A)\inv B)$ angewendet, um die Anzahl der Eigenwerte zu bestimmen.
            Genau dies ist auch die Grundidee in \cite{grundlageFutamura,hauptteilTkachuk}.
% hier noch entscheiden, was drin gelassen wird

% falls man Schur-Zerlegung auf S anwendet, um die Eigenwerte zu sortieren, dann muss man auch gleiche Ähnlichkeitstransformation auf T anwenden, aber ist es dann noch eine obere Dreiecksmatrix?
            
\chapter{Massen- und Steifigkeitsmatrizen}
\label{sec: MS Matrizen}
      Die Berechnung von Eigenkreisfrequenzen hat in der Mechanik einen besonderen Stand:
      Durch die Betrachtung von Systemen und der Berechnung von Eigenfrequenzen, kann vorhergesagt werden, mit welchen Frequenzen das System schwingen wird.

      Durch das kontrollierte Verändern eines Design-Parameters $s$ kann sichergestellt werden, dass keine Eigenfrequenz des Systems in einem vorgegebenen Intervall liegt.
      Durch die Eliminierung von allen Eigenwerten aus diesem Intervall wird garantiert, dass keine Frequenz in diesem Intervall das System anregt.
      Es kommt daher zu keiner Interferenz und Überlagerung, was für große mechanische Systeme katastrophal sein kann.

      In dieser Ausarbeitung werden die folgenden Systeme untersucht:
      \begin{figure}[ht]
            \centering
            \begin{minipage}[ht]{0.49\linewidth}
                  \centering
                  \includegraphics[width=0.9\textwidth, keepaspectratio]{./System1.png}
                  \caption{System 1}
                  \label{fig: System 1}
            \end{minipage}
            \hfill
            \begin{minipage}[ht]{0.49\linewidth}
                  \centering
                  \includegraphics[width=0.9\textwidth, keepaspectratio]{./System2.png}
                  \caption{System 2}
                  \label{fig: System 2}
            \end{minipage}
      \end{figure}

      Der Vektor $x = (x_1,\dots,x_n)^T$ enthält hier die Auslenkungen der Massen aus ihrer Ruhelage und $s\in\R^l$ wird Design-Parameter (vgl. \cite[S. 2]{hauptteilTkachuk}) genannt, da die Eigenschaften des Systems, wie die Eigenkreisfrequenzen, von ihm abhängen.

      Die Systeme aus Abb. \ref{fig: System 1} und \ref{fig: System 2} werden \zitat{Schwingerketten}\cite[S. 236]{maschinendynamikDresig} genannt, da sie aus starren Massen und masselosen Federn bestehen, die linear miteinander verbunden sind.
      Es existieren zudem in diesen Systemen keine Dämpfung.
      
      Laut \cite[S. 362-365]{maschinendynamikDresig} könne man für diese Systeme die \zitat{Differentialgleichung der freien Schwingungen}\cite[S. 365]{maschinendynamikDresig} erhalten, wenn man für jede Masse ein Kräftegleichgewicht aufstellt.
      Diese Differentialgleichung besitzt folgende Form:
      \begin{equation}
            \label{eqn: Dgl freie Schwingungen}
            Kx+M\ddot x = 0
      \end{equation}

      wobei $K$ die Steifigkeitsmatrix und $M$ die Massenmatrix des Systems sind.
% bin ich genug auf d
      Da genau diese Matrizen benötigt werden, um die Eigenkreisfrequenzen zu berechnen, ist das Ziel dieses Kapitels die Systeme von Kräftegleichgewichten aufzustellen und anschließend so umzuformen, dass eine Gleichung der Form (\ref{eqn: Dgl freie Schwingungen}) entsteht.
      Anhand dieser Gleichung werden die Matrizen abgelesen.

      Dafür wird der erste Abschnitt die im System wirkenden Kräfte und die Anwendung des Prinzips von d'Alembert behandeln.
      Im zweiten Abschnitt werden die Matrizen der Systeme bestimmt und anschließend wird in Abschnitt \ref{sec: Formel EW} die Formel für die Berechnung der Eigenkreisfrequenzen hergeleitet. 
      
      \section{Herleitung durch das Prinzip von d'Alembert}
            Nach \cite{d_AlembertPrinzip} besage das Prinzip von d'Alembert, dass die Summe aller wirkenden Kräfte in einem beschleunigten System verschwinde.

            Somit wird zuerst ein Kräftegleichgewicht aufgestellt und anschließend in eine Gleichung über Matrizen umgeformt, wodurch man die Massen- und Steifigkeitsmatrix erhält.
% In Maschinendynamik war von Koeffizientenvgl die Rede, könnte man auch einbauen
% Es muss irgendwo eine Quelle geben, die genau das sagt
            Da diese Herleitung auf Kräftegleichgewichten beruht, werden nun die agierenden Kräfte kurz erläutert:

            Die Federkraft \FL wird laut \cite{federkraft} durch
            \begin{equation}
                  \label{eqn: Federkraft}
                  \FL = -c\cdot s
            \end{equation}
            berechnet, wobei $c$ die Federkonstante und $s$ die Auslenkung der Feder aus der Ruhelage ist.

            Beachte, dass die Kraft entgegen der Auslenkung wirkt, da bei Auslenkung die Feder in die Ruhelage zurückkehren will.
% Quelle finden
                  
            Die Trägheitskraft \FT ist nach \cite{trägheitskraft} definiert durch:
            \begin{equation}
                  \label{eqn: Trägheitskraft}
                  \FT = -m\cdot a
            \end{equation}
            mit Masse $m$ und Beschleunigungsvektor $a$.    
            
            Auch hier wirkt die Kraft wieder der Beschleunigung entgegen.

            Um das Schemas der Kräfte klarer zu gestalten, werden die Kraftpfeile immer in die der Auslengung entgegengesetzte Richtung gezeichnet und die Minuszeichen weggelassen,
            somit werden in den Kraftschemas keine negativen Beträge verwendet.
% Schemas, Kraftschemas,...
            Beide Kräfte werden in Abb. \ref{fig: KräfteAnFeder} veranschaulicht.

            \bild{Federkraft und Trägheitskraft.png}{0.3}{wirkende Kräfte an Masse mit Feder}{\label{fig: KräfteAnFeder}}

            In Abb. \ref{fig: KräfteAnFeder} ist $z$ die Auslenkung und demnach gilt $F_L = c\cdot z$ (vgl. \cite{federkraft}).

            Da in dieser Ausarbeitung nur Feder-Masse-Systeme behandelt werden, kann man $s$ und $a$ auf die Auslenkungen $(\xk)_k$ zurückführen:
            Im Allgemeinen wird hier $s$ durch $(x_{k+1}-\xk)$ und $a$, wie schon in Abb. \ref{fig: KräfteAnFeder}, durch die zweite Ableitung $\ddot \xk$ ersetzt.
% Man könnte es auch umschreiben
            Betrachte nun ein System mit 3 Massen, wie es in Abb. \ref{fig: 3 Massen System} dargestellt wird.
            Anhand dieses Systems soll eine allgemeines Kräftegleichgewicht aufgestellt werden.

            \begin{figure}[ht]
                  \centering
                  \begin{minipage}[ht]{0.49\linewidth}
                        \centering
                        \includegraphics[width=0.7\textwidth, keepaspectratio]{./src/3 Massen System.png}
                        \caption{System mit 3 Massen}
                        \label{fig: 3 Massen System}
                  \end{minipage}
                  \hfill
                  \begin{minipage}[ht]{0.49\linewidth}
                        \centering
                        \includegraphics[width=0.7\textwidth, keepaspectratio]{./src/Kräfte Masse 1D.png}
                        \caption{Kräfte an frei geschnittener Masse}
                        \label{fig: Kräfte Masse 1D}
                  \end{minipage}
            \end{figure}

            Für $m_2$ erhält man die wirkenden Kräfte, wie sie in Abb. \ref{fig: Kräfte Masse 1D} dargestellt wurden.

            Mit dem Prinzip von d'Alembert führt dies auf die Gleichung
            \begin{equation}
                  \label{eqn: Gl für Masse}
                  m_2\,\ddot x_2 + a\,(x_2-x_1) - b\,(x_3-x_2) = 0
            \end{equation}

            Nehme man an, man hat Dirichlet-Randbedingungen, also es gelte zum Beispiel $x_1 \equiv 0$, dann würde man aus (\ref{eqn: Gl für Masse})

            \begin{equation}
                  \label{eqn: Gl für Masse mit Rand links}
                  m_2\,\ddot x_2 + a\,x_2 - b\,(x_3-x_2) = m_2\,\ddot x_2 + (a+b)\,x_2 -b\,x_3 = 0
            \end{equation}
            erhalten.

            In (\ref{eqn: Gl für Masse mit Rand links}) kann die erste Masse als Rand betrachtet werden.
            Eine Gleichung dieser Form wird daher in den Systemen aus Abb. \ref{fig: System 1} und \ref{fig: System 2} für die linken Massen gelten.

      \section{Ermittlung der Matrizen der Modellprobleme}
            Berechne nun die Matrizen der Systeme.
            Während die Systeme der Gleichungen allgemein gehalten werden, verwende für die endgültige Definition der Matrizen $K$ und $M$
            $$n=8, j=n/2$$ 
            Beginne dazu mit dem System aus Abb. \ref{fig: System 1}.

            Die Form der Gleichung für die linke Masse ist durch (\ref{eqn: Gl für Masse mit Rand links}) gegeben.
            Für $k=2,\dots, n-1$ gilt eine Gleichung der Form (\ref{eqn: Gl für Masse}).
            
            Da die letzte Masse durch die rechte Feder mit dem Rand verbunden ist, gilt also $x_{n+1} \equiv 0$.
            Und analog zu (\ref{eqn: Gl für Masse mit Rand links}) erhält man
            $$(s+1)\,\ddot x_n + c\,(x_n-x_{n-1}) + c\,x_n = 0.$$  

            Falls man nun alle Massen, Federkonstanten und Auslenkungen einsetzt, so erhält man nach Umstellen das Gleichungssystem

            $$\begin{cases}
                  m\ \ddot x_1 + 2s\,x_1 - s\,x_2 & = 0   \\
                  m\,\ddot x_k -s\,x_{k-1} + 2s\,x_k -s\,x_{k+1} & = 0,\quad k=2,\dots,j-1\\
                  (s+1)\,\ddot x_j -s\,x_{j-1} + (s+c)\,x_j -c\,x_{j+1} & = 0\\
                  (s+1)\,\ddot x_k -c\,x_{k-1} + 2c\,x_k -c\,x_{k+1} & = 0,\quad k=j+1,\dots,n-1\\
                  (s+1)\,\ddot x_n -c\,x_{n-1}+ 2c\,x_n & = 0
            \end{cases}$$

            Dieses System kann man zusammenfassen zu der Gleichung (\ref{eqn: Dgl freie Schwingungen}),
            wobei
            \begin{align}
                  M &= \diag(m, m, m, s+1, s+1,s+1, s+1,  s+1),\label{def: M1}\\
                  K &= \begin{pmatrix}
                        2s & -s &  &  &  &  &  &  \\
                        -s &  2s& -s &  &  &  &0  &  \\
                         & -s & 2s & -s &  &  &  &  \\
                         &  & -s & s+c & -c &  &  &  \\
                         &  &  & -c & 2c & -c &  &  \\
                         &  &  &  & -c & 2c & -c &  \\
                         & 0 &  &  &  & -c & 2c &  -c\\
                         &  &  &  &  &  & -c & 2c \\
                        \end{pmatrix}\label{def: K1}
            \end{align}\\

            Wende dieses Vorgehen auch für das System aus Abb. \ref{fig: System 2} an und erhalte
            
            $$\begin{cases}
                  m\ \ddot x_1 + (s_1+s_2)\,x_1 - s_2\,x_2 & = 0   \\
                  m\,\ddot x_k -s_k\,x_{k-1} + (s_k+s_{k+1})\,x_k -s_{k+1}\,x_{k+1} & = 0,\quad k=2,\dots,j-2\\
                  s_j\,\ddot x_{j-1} -s_{j-1}\,x_{j-2} + (s_{j-1}+c)\,x_{j-1} -c\,x_{j+1} & = 0\\
                  s_j\,\ddot x_k -c\,x_{k-1} + 2c\,x_k -c\,x_{k+1} & = 0,\quad k=j,\dots,n-1\\
                  s_j\,\ddot x_n -c\,x_{n-1}+ 2c\,x_n & = 0
            \end{cases}$$
            

            Dieses System kann man analog zu oben zu der Gleichung (\ref{eqn: Dgl freie Schwingungen}) umformen.
            Hier gilt nun aber
            \begin{align}
                  M &= \diag(m, m, s_4, s_4, s_4, s_4, s_4, s_4)\label{def: M2}\\
                  K &= \begin{pmatrix}
                        s_1+s_2 & -s_2 &  &  &  &  &  &  \\
                        -s_2 &  s_2+s_3& -s_3 &  &  &  &0  &  \\
                              & -s_3 & s_3+c & -c &  &  &  &  \\
                              &  & -c & 2c & -c &  &  &  \\
                              &  &  & -c & 2c & -c &  &  \\
                              &  &  &  & -c & 2c & -c &  \\
                              & 0 & &  &  & -c & 2c &  -c\\
                              &  &  &  &  &  & -c & 2c \\
                        \end{pmatrix}\label{def: K2}
            \end{align}
                  
% hier muss was über die Eigenschaften gesagt werden
      \section{Zusammenhang Eigenwert und Eigenfrequenz}
            \label{sec: Formel EW}      
            Nach der Ermittlung der zugehörigen Matrizen bestimme nun die Formel für die Eigenkreisfrequenzen eines Systems.

            Die Schritte in diesem Abschnitt folgen \cite[S. 380]{maschinendynamikDresig} und \cite[S. 2]{hauptteilTkachuk}.
            Nach \cite[S. 380]{maschinendynamikDresig} gelte für die Auslenkungen der Massen mit $x$ statt $q$:
            $$x(t) = v\,\exp(i\w t),\ \ \ddot x(t) = -\w^2\,v \exp(i\w t)$$
            mit $v=(v_1,\dots,v_n)^T$ Vektor der Amplituden und der \zitat{noch unbekannten Eigenkreisfrequenz \w}\cite[S. 380]{maschinendynamikDresig}
            
            Diesen Ansatz wird in (\ref{eqn: Dgl freie Schwingungen}) eingesetzt und man erhält nach Division durch $\exp(i\w t)\footnote{offensichtlich ist die Division wohldefiniert, da $\exp(i\w t)\ne 0\ \forall \w, t\in\R,$}$:
            
            \begin{equation}
                  \label{eqn: allg EW Problem mit w}
                  (K-\w^2\,M)\,v = 0
            \end{equation}

            Die Gleichung kann ebenfalls in \cite[S. 380]{maschinendynamikDresig} gefunden werden.

            Man will diese Gleichung für $v\ne 0$ lösen, da man sonst eine triviale Lösung erhält, in der keine Masse schwingt.

            Um die Theorie aus Kapitel \ref{sec: EW Problem_Futamura} anzuwenden, so definiere
            $$\lambda_i := \w_i^2,\quad i=1,\dots,n$$

            Es folgt das allgemeine Eigenwertproblem
            \begin{equation}
                  \label{eqn: allg EW Problem mit lam}
                  (\K-\lambda\,\M)\,v = 0
            \end{equation}

            Man beachte, dass diese Gleichung genau die Form aus (\ref{eqn: allg EW Problem}) besitzt.
            Man kann für die vorgestellten Probleme folgern, dass für $N$ aus Kapitel \ref{sec: Futamura}
            $$N=\{1,\dots,n\}$$

            gilt, da die Massenmatrizen aus (\ref{def: M1}) und (\ref{def: M2}) hier Diagonalmatrizen mit positiven Einträgen auf der Diagonalen sind. Somit sind auch alle Eigenwerte positiv.

            Für reguläre Massenmatrizen gelte nach \cite[S. 376]{matrixGolub} für (\ref{eqn: allg EW Problem mit lam}):
            $$(K-\lambda\,M)\,v = 0 \Leftrightarrow (M\inv K-\lambda\In)\,v=0$$
            und folglich
            $$\lambda(K, M) =\lambda(M\inv K,\In) = \lambda(M\inv K)$$

            Somit sind die Eigenwerte des Matrix-Pencils $K-\lambda M$ und der Matrix $M\inv K$ äquivalent.

            Da auch $M\inv$ eine Diagonalmatrix ist, so ist $M\inv K$ für die vorgestellten Systeme symmetrisch.
            Es gilt daher für die Systeme aus Abb. \ref{fig: System 1} und \ref{fig: System 2}
            \begin{equation}
                  \label{eqn: alle Ew reell}
                  \lambda(K, M) = \lambda(M\inv K) \subset \R
            \end{equation}
            wobei \cite[S. 393]{matrixGolub} angewendet wurde.

            Durch Bemerkung \ref{bem: B reg impl pencil reg} folgt ferner, dass die Matrix-Pencil der Systeme regulär sind.

\chapter{Zählung von Eigenwerten}
\label{sec: EW Zählung}
% vielleicht eher Finden der Zielfunktion nennen
% es fehlt noch: Einführung s und Erklärung, warum nun lambda statt w^2
      Nun wende man sich dem Finden einer Zielfunktion zu, die die Anzahl der Eigenwerte auf einem Intervall beschreibt.
      Diese Funktion hängt von dem Design-Parameter $s$ (vgl. \cite[S. 2]{hauptteilTkachuk}) ab und wird offenbar minimal, wenn kein Eigenwert des Matrix-Pencils in dem Intervall liegt.
      
      \section{Vorüberlegungen}
            Die in diesem Abschnitt erwähnten Schritte stammen aus \cite[S. 2-4]{hauptteilTkachuk}.

            Betrachte Gleichung (\ref{eqn: allg EW Problem mit lam}), durch die Definition $\lam = \w^2$ verändert sich auch das Intervall, welches man betrachtet:
            $$\lamAlamB:=\wAwB$$
% \w_a und \w_b nicht erklärt
            Definiere nun die Funktion $h$ wie folgt:
            $$h(\lambda):=\1_{[\lambda_a,\lambda_b]}(\lambda)$$

            Da man nur daran interessiert ist, ob ein Eigenwert in dem Intervall liegt oder nicht, definiere

            $$\mu := \sum_{j\in N} h(\lambda_j)$$

            Nach Kapitel \ref{sec: Formel EW} gilt aber $N=\{1,\dots,N\}$.\\
            Obwohl diese Funktion die Anzahl der Eigenwerte korrekt beschreibt, gibt es einige Schwachstellen, sollte man diese Funktion minimieren wollen:
            Die Funktion ist eine Treppenfunktion, damit ist sie nicht stetig und alle Ableitungen sind gleich 0, also unbrauchbar für Minimierungsalogrithmen, die eine Ableitung verwenden.

            Diese Probleme kann man abfedern, indem man die Funktion auf dem Intervall $[\lambda_a, \lambda_b]$ gewichtet.
% abfedern
            Nutze dazu eine Gewichtungsfunktion, die auf dem ganzen Intervall positiv und konkav ist.
            Diese Funktion sorgt dafür, dass Eigenwerte in der Mitte des Intervalls stärker ins Gewicht fallen
            Nach \cite[S. 3]{hauptteilTkachuk} sei die Funktion
            $$g(z) = -(z-((1+\alpha)\lambda_a -\alpha\lambda_b))(z-((1+\alpha)\lambda_b-\alpha\lambda_a))$$
            sehr geeignet, man könnte aber auch andere Funktionen definieren, die diese Eigenschaften besitzen.
            In den Funktionen stellt $\alpha>0$ einen Parameter dar, der dafür sorgt, dass die Funktion auf \lamAlamB positiv ist.
            $\alpha$ wird auch Inflationsparameter (vgl. \cite[S. 3]{hauptteilTkachuk}) genannt.
% Inflationsparameter

            Man findet so die erste Funktion, welche man mit Verfahren der Optimierung sinnvoll minimieren könnte:
            $$J(s) = \sum_{j=1}^n g(\lambda_j)h(\lambda_j)$$

            Hier müsste man allerdings jeden Eigenwert zuerst berechnen, um dann diese Formel anzuwenden.
            Auch hängen weder $g$ noch $h$ explizit von $s$ ab, weshalb diese Funktion eher zur Bestimmung der aktuellen gewichteten Anzahl von Eigenwerten gesehen werden kann, als die tatächliche Zielfunktion, die es zu minimieren gilt.
            Das Ziel ist daher eine Funktion, deren Polstellen die Eigenwerte des Matrix-Pencils sind.
            Diese Funktion kann dann durch ein Kurvenintegral integriert werden, um die Anzahl an Eigenwerten zu bestimmen

            Die gesuchte Funktion sei
            \begin{equation}
                  \label{eqn: ZielfunktionMitPolstellen}
                  L(z, s) = g(z)\sum_{j\in N} \frac{1}{z-\lam_j} = \sum_{j=1}^n \frac{g(z)}{z-\lam_j}
            \end{equation}

            Und das entsprechende Integral ist wie folgt definiert:
            \begin{equation}
                  \label{eqn: IntZielfunktionMitPolstellen}
                 \int_\gamma L(z, s)\ dz
            \end{equation}

            Da alle Eigenwerte auf der reellen Achse liegen, sei $\gamma$ einfachheitshalber ein Kreis $C$.

            Allerdings muss zunächst gezeigt werden, dass diese Gleichung tatsächlich die Eigenwerte in dem Intervall zählt, dafür benötigt man den Residuensatz, welcher im nächsten Abschnitt vorgestellt wird.
      \section{Komplexe Analysis und der Residuensatz}
            Dieses Kapitel behandelt ausschließlich den Residuensatz und die Argumente, warum man ihn auf (\ref{eqn: IntZielfunktionMitPolstellen}) anwenden kann.
            Obwohl $L$ durch die Verteilung der Eigenwerte auch von dem Design-Parameter $s$ abhängt, vernachlässige diese Abhängigkeit in diesem Abschnitt.

            Zuerst wird der Residuensatz und anschließend die benötigte Theorie vorgestellt und auf das Problem oben angewendet.

            \begin{theorem}(Theorem 9.2 (Residue theorem)\cite[S. 141]{complexAnalysis})\\
                  \label{thrm: Residuensatz}
                  Sei $f$ analytisch auf $\Omega$, mit Ausnahme der isolierten Singularitäten $a_1,\dots,a_n$. Wenn für einen Kreis $\gamma\subseteq\C$ die Bedingungen $\gamma \sim 0$ und $a_j\notin \gamma,\ j=1,\dots,n$ erfüllt sind, dann
                  $$\int_\gamma f(z)\ dz = 2\pi i\sum_{k=1}^{n} n(\gamma, a_k)\Res_{a_k}f$$
            \end{theorem}
            Beweis: siehe \cite[S. 142]{complexAnalysis}

            Die folgenden Aussagen erklären die Voraussetzungen und das Resultat des Theorems und stammen alle aus \cite{complexAnalysis}.
            Die Resultate und Definitionen werden nur kurz angeschnitten, da man sie in dieser Ausarbeitung nur für den Residuensatz \ref{thrm: Residuensatz} verwendet.
            \begin{itemize}
% nicht alle sind fett markiert
                  \item \textbf{analytisch in }$\mathbf{z:}$ ist eine Eigenschaft von komplexen Funktionen, die besagt, dass man die Funktion als eine Potenzreihe um $z$ entwickeln kann.
                        Für (\ref{eqn: ZielfunktionMitPolstellen}) reicht aber folgende Aussage (vgl. \cite[S. 24]{complexAnalysis}):
% man müsste die Funktion referenzieren
                        Für Polynome $p(z)$ und $q(z)$ sei die rationale Funktion $r(z):=\frac{p(z)}{q(z)}$ analytisch auf $\{z\in\C: q(z)\ne 0\}$.
                        
                        Offenbar ist (\ref{eqn: ZielfunktionMitPolstellen}) eine rationale Funktion:

                        $$L(z) = \frac{p(z)}{q(z)}$$
                        mit 
                        \begin{align*}
                              p(z) &= \sum_{j=1}^{n}\prod_{\underset{i\ne j}{i=1}}^{n} z-\lambda_i,\\
                              q(z) &= \prod_{i=1}^{n} z-\lambda_i
                        \end{align*}

                        Offenbar sind $p$ und $q$ Polynome und $\{z: q(z)\ne 0\} = \C\setminus\{\lambda_i, i=1,\dots,n\}$
                        Daher ist $L$ analytisch auf $\C$ bis auf $\lambda_1,\dots,\lambda_n$
                  \item $\mathbf{b}$\textbf{ isolierte Singularität}:
                        Laut \cite[S. 74]{complexAnalysis} habe $f$ eine isolierte Singularität bei $b$, wenn $f(b)$ nicht definiert und$f$ analytisch für $0<|z-b|<\epsilon$ mit $\epsilon>0$ sei.

                        Offenbar ist $L(\lambda_i)$ nicht definiert für $i=1,\dots,n$ und da $L$ nur eine endliche Anzahl an Singularitäten besitzt, gilt:

                        \begin{equation}
                              \label{hilfe: complexAnalysis_isolierteSingularitäten}
                              L \text{ analytisch für } 0<|z-\lambda_i|<\widetilde{\epsilon},\quad i=1,\dots,n
                        \end{equation}
                        mit
                        $$\widetilde{\epsilon}:= \frac{1}{2}\,\min_{i,j\in N: \lambda_i\ne \lambda_j} |\lambda_i-\lambda_j|$$
% alternative: \min_{\lambda_i,\lambda_j\in\lambda(K,M): \lambda_i\ne\lambda_j}
                        Für den Fall, dass ein Eigenwert doppelt vorkommt, so ist der Eigenwert trotzdem eine isolierte Singularität, da die Definition dies zulässt.
% aus den Fingern gesaugt
                  \item $\frac{1}{2\pi i}\int_{C} \frac 1 {z-a}\ dz:$
                        Dieses Integral ist zwar nicht direkt für Theorem \ref{thrm: Residuensatz} von Bedeutung, allerdings wird damit vieles vereinfacht.

                        Sei $z_0\in\C,\ r>0\text{ und }C=\{z_0+r e^{it}: t\in [0,2\pi]\}$ (vgl. \cite[S. 48]{complexAnalysis}), laut Proposition 4.13 aus \cite[S. 48]{complexAnalysis} gelte:
                        \begin{equation}
                              \label{hilfe: complexAnalysis_IntegralEinsDurchX}
                              \frac{1}{2\pi i}\int_C \frac 1 {z-a}\ dz = \begin{cases}
                                    1 & \text{: } |a-z_0|<r \\
                                    0 & \text{: } |a-z_0|>r
                                    \end{cases}
                        \end{equation}
                  \item $n(\gamma, a)$:
                        Sei $\gamma$ ein Kreis und $a\in \C\setminus \gamma$.
                        Nach Definition 5.4 aus \cite[S. 65]{complexAnalysis} ist die Windungsnummer $n(\gamma,a)$ von $\gamma$ über $a$ wie folgt definiert:
                        $$n(\gamma,a):= \frac{1}{2\pi i}\int_\gamma \frac{1}{\xi-a}\ d\xi$$

                        Man erkennt, dass mit $\gamma = C$  und (\ref{hilfe: complexAnalysis_IntegralEinsDurchX}) gilt\footnote{$C$ aus (\ref{hilfe: complexAnalysis_IntegralEinsDurchX}) gemeint}:
                        \begin{equation}
                              \label{hilfe: complexAnalysis_WindungNullEins}
                              n(\gamma,a) = \begin{cases}
                                    1& \text{ : }a\in B_r(z_0)\\
                                    0& \text{ : }a\notin\overline {B_r(z_0)}
                              \end{cases}
                        \end{equation}
                        Hierbei benötigt man das $C$ aus (\ref{hilfe: complexAnalysis_IntegralEinsDurchX}), damit der Kreis in positive Richtung durchlaufen wird und kein Punkt (bis auf $z=z_0+r$) mehrmals vorkommt.
% jetzt klingt es so, also könne kein anderes C dies vollbringen
                  \item $\gamma \sim 0$:
                        Sei $\gamma$ geschlossene Kurve und $\Omega$ ein Gebiet, durch Anwenden von Definition 5.5 aus \cite[S. 67]{complexAnalysis} folgt:
                        $$\gamma \sim 0 \Leftrightarrow n(\gamma, a)=0\ \forall a\notin \Omega$$

                        Für das obige Problem sei $\Omega = \C$, offenbar gilt damit für alle $\alpha \in \Omega^C$
                        $$n(\gamma, \alpha) = 0$$

                  \item $\Res_af$:
                        Nach Beispiel 9.3 aus \cite[S. 142]{complexAnalysis} gilt für eine einfache Polstelle:
                        \begin{equation}
                              \label{hilfe: complexAnalysis_Residuum}
                              \Res_af = \lim_{z\to a} (z-a)f(z)
                        \end{equation}
                        Anwenden auf $\Res_{\lambda_i}L$ ergibt:
                        \begin{align*}
                              \Res_{\lambda_i}L =& \lim_{z\to \lambda_i} \sum_{j=1}^{n} (z-\lambda_i)\frac{g(z)}{z-\lambda_j} = g(z)\underbrace{\sum_{\underset{j\ne i}{j=1}}^{n} \frac{z-\lambda_i}{z-\lambda_j}}_{\to 0} + g(z)\frac{z-\lambda_i}{z-\lambda_i} \\
                              =& g(\lambda_i)
                        \end{align*}

                        Man beachte, dass für einen Eigenwert $\lambda$, der k-mal vorkommt, gilt:
                        $$\Res_\lambda L = k$$
                        Am Ende erhält man hier immer die Anzahl an Eigenwerten, die sich in dem Integrationsgebiet befinden.
            \end{itemize}

            Damit wurde gezeigt, dass der Residuensatz auf die Funktion $L$ angewendet werden kann.

% kann auch rausgelassen werden
            Ferner gelte nach \cite[S. 77]{complexAnalysisVL} der Residuensatz auch für \zitat{toy contours}\cite[S. 77]{complexAnalysisVL}.
% Zitat überprüfen/ übersetzen
            Beispiele für toy contours sind Rechtecke oder Kreisausschnitte (siehe \cite[S. 42]{complexAnalysisVL}).

      \section{Anwendung des Residuensatzes und der Identität von Futamura}
            Mit dem Residuensatz aus Theorem \ref{thrm: Residuensatz} gilt folglich für (\ref{eqn: IntZielfunktionMitPolstellen}):
            $$\frac 1 {2\pi i}\int_\gamma L(z,s)\ dz = \sum_{k=1}^{n} n(\gamma, \lambda_k) \Res_{\lambda_k}L = \sum_{k=1}^{n} n(\gamma, \lambda_k) g(\lambda_k)$$

            mit \begin{align}
                  \gamma =& \{\Tilde{z_0} + \Tilde r\exp(i t): t\in [0,2\pi]\},\label{def: Kreis}\\
                  \Tilde{z_0} =& \frac {\lambda_a+\lambda_b} {2},\quad\Tilde r = \frac {\lambda_b-\lambda_a} {2}\nonumber
            \end{align}

            Mit (\ref{hilfe: complexAnalysis_WindungNullEins}) folgt für $n(\gamma, \lam_k)$:
            $$n(\gamma, \lam_k) = \begin{cases}
                  1& \text{ : }\lam_k\in B_{\Tilde{r}}(\Tilde{z_0})\\
                  0& \text{ : }\lam_k\notin\overline {B_{\Tilde{r}}(\Tilde{z_0})}
            \end{cases}$$
            
            Da aber $\lam_k\in \R\ \forall k=1,\dots,n$, ist diese Gleichung äquivalent zu
            $$n(\gamma, \lam_k) = \begin{cases}
                  1& \text{ : }\lam_k\in (\lam_a,\lam_b)\\
                  0& \text{ : }\lam_k\notin \lamAlamB
            \end{cases} = \1_{\lamAlamB}(\lambda_k) = h(\lambda_k)$$

            Beachte, dass für den Fall $\lam_k \in \{\lam_a, \lam_b\}$ diese Theorie und auch der Residuensatz aus Theorem \ref{thrm: Residuensatz} keine Aussage trifft, dieser Fall tritt aber fast sicher nicht ein.
            Für (\ref{eqn: IntZielfunktionMitPolstellen}) folgt
            $$\frac 1 {2\pi i}\int_\gamma L(z,s)\ dz = \sum_{k=1}^{n} h(\lambda_k) g(\lambda_k) = J(s)$$
            
            und mit der Identität von Futamura:
            \begin{align}
                  J(s) =&\, \frac 1 {2\pi i}\int_\gamma L(z,s)\ dz = \frac 1 {2\pi i}\int_\gamma g(z)\sum_{j\in N} \frac{1}{z-\lam_j}\ dz\nonumber\\
                  =&\, \frac 1 {2\pi i}\int_\gamma g(z)\tr((zM(s)-K(s))\inv M(s))\ dz\label{eqn: JGleichIntTr}
            \end{align}

            Diese Funktion hängt nun explizit von $s$ ab und kann daher in Kapitel \ref{sec: Programmieren} minimiert werden.
            
            Da diese Funktion in Kapitel \ref{sec: Programmieren} mithilfe des Gradientenverfahrens minimiert wird, benötigt man zusätzlich die Ableitung
            \begin{equation}
                  \label{eqn: AbleitungZielfunktion}
                  \frac{\partial L(z,s)}{\partial s} = g(z) \frac{\partial}{\partial s}\klammer{\tr(\underbrace{(zM-K)\inv M}_{=:B(z, s)})}
            \end{equation}

            und es gilt für $\frac{\partial}{\partial s} \klammer{\tr(B(z, s))}$:
            $$\frac{\partial}{\partial s} \klammer{\tr(B(z, s))} = \frac{\partial}{\partial s} \klammer{\sum_{i=1}^n (B(z, s))_{ii}} = \sum_{i=1}^n \frac{\partial}{\partial s} (B(z,s))_{ii} = \tr\klammer{\frac{\partial}{\partial s} B(z,s)}$$.

            Beachte, dass für $s\in \R^l$, $z$ fest und $B\in \R^{n\times n}$ gilt:

            $$B(z, \cdot): \R^l \to \R^{n\times n}\Rightarrow \frac{\partial}{\partial s} B(z, s) : \R^l\to \R^{n\times n}\times \R^l \Rightarrow \tr\klammer{\frac{\partial}{\partial s} B(z, s)}: \R^l \to \R^l$$

            Man kann sich $\frac{\partial}{\partial s} B(z, s)$ als eine $n\times n$-Matrix vorstellen, in der jeder Eintrag ein Vektor in $\R^l$ ist.
            Daher summiert man auch mit der Spur die Vektoren in $\R^l$ auf der Diagonalen auf und erhält selber wieder einen Vektor in $\R^l$.
% könnte man auch weglassen
            
            Ferner gilt für $\frac{\partial}{\partial s} B(z, s)$:
            \begin{align*}
                  \frac{\partial}{\partial s} B(z, s) =& \frac{dD}{ds} \ M + D \ \frac{dM}{ds}\\
                  =& - D \frac{d}{ds}(zM-K) D + D \ \frac{dM}{ds}\\
                  =& D \ \frac{dM}{ds} - D \klammer{z \frac{dM}{ds}-\frac{dK}{ds}} D
            \end{align*}

            Wobei der Lesbarkeit halber $D:=(zM-K)\inv$ gilt und im 2. Schritt die Formel
            $$\nabla A(t)\inv = -A(t)\inv\ \nabla A(t)\ A(t)\inv$$
            angewendet wurde.\\
            Beweis (vgl. \cite{derivativeInverseMatrix}):
            Seien $a_{ij}(t)$ die Elemente der Matrix $A(t)$ und $\widetilde{a}_{ij}(t)$ die Elemente von $A\inv(t)$.

            Dann gilt für alle $t$:
            $$\sum_{j=1}^{n} a_{ij}(t)\widetilde{a}_{jk}(t) =  \begin{cases}
                        1 & \text{ :  }\, i=k \\
                        0 & \text{ :  sonst } 
                        \end{cases} = \text{kons. für }t $$
            Folglich gilt für die Ableitung:
            $$\frac{d}{dt}\sum_{j=1}^{n} a_{ij}(t)\widetilde{a}_{jk}(t) = \sum_{j=1}^{n} \frac{d}{dt}a_{ij}(t)\ \widetilde{a}_{jk}(t)+a_{ij}(t)\ \frac{d}{dt}\widetilde{a}_{jk}(t) = 0 \quad \forall i,k\in\{1,\dots,n\}$$
            Diese Gleichungen kann man in eine Gleichung über Matrizen umformen:
            $$\frac{dA}{dt}\ A\inv + A\ \frac{dA\inv}{dt} = 0$$
            Nach Umstellen folgt damit die Behauptung.\qed
% in Quelle wurde t auf ein Intervall beschränkt, A muss Inverse haben und stetig diffb. sein für alle t, das müsste man noch zeigen

            Die endgültige Formel für die Ableitung der approximierten gewichteten Eigenwert-Zählung sieht folgendermaßen aus:
            \begin{align}
                  \label{eqn: vollständigeZielfunktion}
                  \nabla J^*(s) =& \frac 1 {2\pi i}\sum_{k=0}^{m-1} g(z) \tr\klammer{D \ \frac{dM}{ds} - D \klammer{z \frac{dM}{ds}-\frac{dK}{ds}} D}\,c_k\\
                  =& \frac 1 {2\pi i}\sum_{k=0}^{m-1} g(z) \klammer{\tr\klammer{D \ \frac{dM}{ds}} - \tr\klammer{D \klammer{z \frac{dM}{ds}-\frac{dK}{ds}} D}}\,c_k
            \end{align}
% man könnte noch auf die Wohldefiniertheit eingehen, da es mit den Dimensionen passt, aber es geht erstmal
            
\chapter{Implementierung}
\label{sec: Programmieren}
      \section{Erklärungen}

      \section{Implementation}

      \section{Grafiken}

\chapter{Verbesserungen}
\label{sec: Verbesserungen}
      \section{Auswertung bisheriger Implementation}

      \section{mögliche Verbesserungen}

      \section{Implementation und Auswertung der verbesserten Programme}

      \section{Approximation der Spur}

\chapter{Auswertung}
\label{sec: Auswertung}

\chapter{Literaturverzeichnis}

      \printbibliography
% es fehlt Bilderverzeichniss

%%
%% Erscheint auf letzter Seite
%%
\chapter*{Erkl\"{a}rung}
\thispagestyle{empty}
Hiermit erkl\"{a}re ich, dass ich die am \datum\ eingereichte Bachelorarbeit zum Thema
\emph{\thema} unter Betreuung von \betreuer\ selbstst\"{a}ndig erarbeitet,
verfasst und Zitate kenntlich gemacht habe. Andere als die angegebenen Hilfsmittel
wurden von mir nicht benutzt.

\bigskip \bigskip \bigskip \bigskip \bigskip

Dresden, \datum\ \hfill Unterschrift

\normalsize
\end{document}



