\documentclass[ngerman,BCOR=4mm]{tudscrreprt}

\usepackage[T1]{fontenc}
\usepackage[ngerman=ngerman]{hyphsubst}
\usepackage[ngerman]{babel}
\usepackage{isodate}
\usepackage[hidelinks]{hyperref}
\usepackage{amsmath, amsfonts, amssymb, amsthm} %% mathematics tools
\usepackage{csquotes}
\usepackage[draft]{listofsymbols}
\usepackage[backend=biber, style=numeric, urldateusetime=true]{biblatex} %% Literature citing engine
% \usepackage{tocbasic}
% \usepackage{floatbytocbasic}

%__________Definitions_____________________________

% Literaturverzeichnis
\addbibresource{Literatur.bib}

\newcommand{\bild}[2]{\includegraphics[width=#2\textwidth,height=#2\textheight,keepaspectratio]{#1}}
\newcommand{\R}{\mathbb R}
\newcommand{\C}{\mathbb C}

% wird für Symbolverzeichnis benutzt, sonst gibt es Fehler
\DeclareOldFontCommand{\bf}{\normalfont\bfseries}{\mathbf}


% Math
\theoremstyle{plain} % text is cursive
\newtheorem{theorem}{Theorem}
\newtheorem{lemma}[theorem]{Lemma}  %% [theorem] means same numbering for theorem and lemma
\newtheorem{proposition}[theorem]{Proposition}
\newtheorem{corollary}[theorem]{Korollar}

\theoremstyle{definition} % text is "upright"
\newtheorem{definition}[theorem]{Definition}
\newtheorem{example}[theorem]{Beispiel}

\theoremstyle{remark}
\newtheorem{remark}[theorem]{Bermerkung}

% Symbolverzeichnis
\renewcommand{\symheadingname}{Symbolverzeichnis}
\opensymdef
      % \newsym[]{}{}
      \newsym[Massenmatrix]{M}{M}
      \newsym[Steifigkeitsmatrix]{K}{K}
      \newsym[Federkraft]{FL}{\overrightarrow{F_L}}
      \newsym[Trägheitskraft]{FT}{\overrightarrow{F_T}}
      \newsym[Verschiebung des k-ten Massepunktes aus der Ruhelage]{xk}{x_k}
      \newsym[Eigenfrequenz]{w}{\omega}
\closesymdef


\begin{document}
\selectlanguage{ngerman}

%__________Deckblatt_Allg_Infos_____________________________
\faculty{Fakultät Mathematik}
\institute{Institut für Numerische Mathematik}
% \chair{Professur für Numerik partieller Differentialgleichungen}
\date{23.08.2024}
\title{%
      Gewichtete Eigenwert-Zählung auf einem Intervall
}
\subject{bachelor}
\graduation[B.Sc.]{Bachelor of Science}
\author{%
      Noah Göpel%
      \matriculationnumber{5029415}%
      \dateofbirth{01.02.2003}%
      \placeofbirth{Riesa}%
}
\matriculationyear{2021}
% \supervisor{}
\professor{Prof. Dr. Oliver Sander}
\maketitle

\clearpage
\tableofcontents
\clearpage
\listofsymbols
\clearpage

\chapter{Einleitung}
\label{sec: Einleitung}
      
      Diese Ausarbeitung beschäftigt sich mit numerischen Verfahren, um sicherzustellen, dass für ein gegebenes mechanisches System keine Eigenwerte in einem vorgegebenem festen Intervall liegen.
% hier muss noch einiges getan werden
      Damit kann sichergestellt werden, dass die Eigenfrequenzen des Systems nicht so liegen, dass es zu einer Selbsterregung kommt.

      Um diese Anforderung zu erfüllen, werden die Eigenwerte des Matrix Pencils auf diesem Intervall gewichtet gezählt. Man erhält ein Minimierungsproblem,
      in welchem es gilt, einen Design-Paramerter so anzupassen, dass auf dem Intervall kein Eigenwert des entsprechenden Systems mehr vorhanden ist.
      Dazu werden in den Kapiteln \ref{sec: MS Matrizen} und \ref{sec: EW Problem} die theoretischen Grundlagen gelegt.
      Anschließend werden in Kapitel \ref{sec: Quellen} die entscheidende Identität von Futamura und wichtige Überlegungen ausgeführt.

      Daraufhin werden diese Überlegungen in Kapitel \ref{sec: Programmieren} anhand verschiedener Beispiele implementiert und getestet.

      Die Auswertung und Verbesserung der Implementation folgt in Kapitel \ref{sec: Verbesserungen}.

      % In dieser Ausarbeitung werden die gewichtete Zählung von Eigenwerten auf einem Intervall behandelt, um sicherzustellen,
      % dass bei einem vorgegebenem mechanischem System keine Eigenwerte in einem bestimmten festen Intervall liegen.
      % Dazu wird in den ersten Kapiteln die nötige Theorie anhand der Quellen \cite{grundlageFutamura} und \cite{hauptteilTkachuk} beschrieben.
      % Hier werden Matrix Pencil und das verallgemeinerte Eigenwertproblem verwendet, um 
      % Hierzu nutzt man die Identität von Futamura, um die Zählung der Eigenwerte auf die zugrundeliegenden Matrizen zurückzuführen.


      % und anschließend in Kapitel \ref{sec: Programmieren} in Python angewendet.
      % Daraufhin werden in Kapitel \ref{sec: Verbesserungen} die Ergebnisse kritisch betrachtet und weitere Algorithmen eingeführt, um die Berechnungen zu beschleunigen.
      Zum Schluss werden in Kapitel \ref{sec: Auswertung} die Ergebnisse wiederholt und ein Ausblick in weiterführende Themen gegeben.

      Ferner werden in Kapitel \ref{sec: Quellen} eine Approximation der Matrix-Spur vorgestellt und ebenfalls implementiert.

\chapter{Massen- und Steifigkeitsmatrizen}
\label{sec: MS Matrizen}
      Um die Eigenfrequenzen und/oder Eigenkreisfrequenzen \w eines Systems zu bestimmen, benötigt man zuerst die Massenmatrix \M und die Steifigkeitsmatrix \K des Systems.

      Laut \cite[S. 366]{maschinendynamikDresig} könne man für einfache Systeme das Prinzip von d'Alembert anwenden, es sei aber ungeeignet für komplexere Probleme.
            
      In diesem Kapitel werden 2 Systeme vorgestellt und die entsprechenden Matrizen berechnet:
      das erste ist ein Längsschwinger-System in einer Dimension, das zweite ein Längsschwinger in 2 Dimensionen.

      Hier folgen nun Bilder, um die Systeme zu veranschaulichen:

      Bild mit Längsschwinger in 1D

      Bild mit Längsschwinger in 2D

      Hierbei haben die Federn keine Massen und es existieren keine Dämpfung oder Gewichtskraft.

      \section{Herleitung durch das Prinzip von d'Alembert}
            Nach \cite{d_AlembertPrinzip} besage das Prinzip von d'Alembert, dass die Summe aller wirkenden Kräfte in einem beschleunigten System verschwinde.

            Somit wird zuerst ein Kräftegleichgewicht aufgestellt und anschließend in Matrix-Schreibweise umgeformt, wodurch man die Massen- und Steifigkeitsmatrix erhält.
% In Maschinendynamik war von Koeffizientenvgl die Rede, könnte man auch einbauen
            Die erhaltene Gleichung muss die Form $M\,\ddot x+K\,x = 0$ besitzen. Dadurch können die gesuchten Matrizen an der Gleichung abgelesen werden.
% Es muss irgendwo eine Quelle geben, die genau das sagt
            Da diese Herleitung auf Kräftegleichgewichten beruht, werden nun die agierenden Kräfte kurz erläutert:

            Die Federkraft \FL wird laut \cite{federkraft} berechnet durch:
            \begin{equation}
                  \label{eqn: Federkraft}
                  \FL = -k\cdot \overrightarrow s,
            \end{equation}
            wobei $k$ die Federkonstante und $\overrightarrow s$ die Auslenkung der Feder aus der Ruhelage darstellt.

            Beachte, dass die Kraft entgegen der Auslenkung wirkt, da bei positiver Auslenkung die Feder sich zusammenziehen will, die Kraft also in die entgegengesetzte Richtung wirkt.
% Quelle finden
                  
            Die Trägheitskraft \FT ist definiert durch:
% Quelle
            \begin{equation}
                  \label{eqn: Trägheitskraft}
                  \FT = -m\cdot \overrightarrow{a}
            \end{equation}
            mit Masse $m$ und Beschleunigungsvektor $\overrightarrow{a}$.                  

            Um die Schematas klarer zu gestalten, werden die Kraftpfeile immer in die entgegengesetzte Richtung eingezeichnet, dafür aber die Minuszeichen weggelassen,
            somit werden in den Kraftschemas keine negativen Richtungen verwendet.
% Schemas, Kraftschemas, ...

            Generell sind $\FL,\ \overrightarrow{s},\ \FT \text{ und } \overrightarrow{a}\in \R^n$, aber in dieser Ausarbeitung werden sie immer als Skalare verwendet. 
            Da in diesem Kapitel Feder-Masse-Systeme behandelt werden, kann man $\overrightarrow{s}$ und $\overrightarrow{a}$ auf die Verschiebungen der Massen $(\xk)_k$zurückführen.
            Im Allgemeinen wird $\overrightarrow{s}$ die Differenz $x_{k+1}-\xk$ und $\overrightarrow{a}$ die zweite Ableitung $\ddot \xk$ sein.
            Beide Kräfte werden in dem folgenden Schema veranschaulicht:

                  
            \begin{figure}[htp]
                  \centering
                  \bild{"./src/KräfteAnFeder.png"}{0.3}
                  \label{fig: KräfteAnFeder}
                  \caption{Kräfte an einer Feder}\cite{federkraft}
            \end{figure}
            


            betrachte ein Beispiel mit 2 Massen, berachte die wirkenden Kräfte für erste Masse.

            Nun kann man dies verallgemeinern auf ein System mit n Massen, wie in Bild

            Es folgen die Gleichungen:



            Man hat somit ein System von Kräftegleichgewichten:

            Dieses kann man als
            \begin{equation}
                  \label{eqn: MK Gleichung in 1D}
                  M\,x^{\dot\dot} + K\,x = 0
            \end{equation}

            Man sieht, dass hier M eine Diagonalmatrix ist, die nur positive Einträge auf der Hauptdiagonalen besitzt.
            Sie ist somit insbesondere positiv definit, da alle Minoranten positiv sind.

            Für den zwei-dimensionalen Fall benötigt man $n^2$ Punkt Massen und für jeden Punkt auch 2 Einträge in $x$, da es nun eine Bewegung in x- und in y-Richtung gibt.
            Somit existieren $2\,n^2$ Einträge in $x$, aber jede Masse besitzt nur 2 Verbindungen mehr, somit wird die Steifigkeitsmatrix \K viel größer, aber auch viel schwächer besetzt.
            
      \section{Ermittlung der Matrizen der Modellprobleme}
            Nun können wir uns mit der Ermittlung der Massen- und Steifigkeitsmatrizen für die Modellprobleme vornehmen.
            Beginne dazu mit dem System in einer Dimension (Bild ....)

            Nach Bsp .... kann man die Gleichung für $m_1$ und $m_n$ übernehmen, für die inneren Punkte gelten folgende Kräfte:

            Bild Kräfte innerer Punkt,

            was auf die Gleichung
            $$m_i\,\ddot x_i + c_{i-1}\,(x_i-x_{i-1}) - c_i\,(x_{i+1}-x_i), \quad i=2,...,n-1$$

            zurückkommt. 
            % Für i=1,n beachte, dass x_0 bzw. x_{n+1} nicht existieren, da diese Massen mit der festen Begrenzung verbunden sind, es folgen somit analog zu Bsp... die Gleichungen für 1 und n

            Nach Umstellen folgt:
            \begin{equation}
                  \label{eqn: System GDgl MK 1d}
                  \begin{cases}
                        m_1\ \ddot x_1 + (c_0+c_1)\,x_1 - c_1\,x_2 & = 0   \\
                        m_i\ \ddot x_i - c_{i-1}\,x_{i-1} + (c_{i-1}+c_i)\,x_i -c_i\,x_{i+1} & = 0,\ i=2,...,n-1 \\
                        m_n\ \ddot x_n - c_{n-1}\,x_{n-1} + (c_{n-1}+c_n)\,x_n & = 0
                  \end{cases}
            \end{equation}

            Dieses System kann man zusammenfassen zu:
            $$M\,\ddot x + K\,x = 0,\quad M= \text{diag}(m_1,\dots,m_n),
            K = \begin{pmatrix}
                  c_0+c_1 & -c_1 &  0& ... & 0 \\
                  -c_1 & c_1+c_2 & -c_2 & \ddots & \vdots \\
                   0& -c_2 & c_2+c_3 & \ddots & 0 \\
                   \vdots & \ddots & \ddots & \ddots  & -c_{n-1} \\
                   0& ... & 0& -c_{n-1} & c_{n-1}+c_n \\
                  \end{pmatrix}$$

      \section{Zusammenhang Eigenwert und Eigenfrequenz}

            wo schreibt man, dass man die Basis $cos(....)....$ nutzt, um auf das Problem $(K-\w^2 M)\, v = 0$ zu kommen?


\chapter{Das verallgemeinerte Eigenwert-Problem}
\label{sec: EW Problem}
      \section{Der Matrix Pencil}

      \section{Das verallgemeinerte Eigenwertproblem}

      \section{Die Schur-Zerlegung}

\chapter{Quellen-Ausarbeitungen}
\label{sec: Quellen}
      \section{Die Identität von Futamura}

      \section{Die Hutchinson-Approximation}

      \section{Die Berechnungen nach Tkachuk}

\chapter{Implementierung}
\label{sec: Programmieren}
      \section{Erklärungen}

      \section{Implementation}

      \section{Grafiken}

\chapter{Verbesserungen}
\label{sec: Verbesserungen}
      \section{Auswertung bisheriger Implementation}

      \section{mögliche Verbesserungen}

      \section{Implementation und Auswertung der verbesserten Programme}

\chapter{Auswertung}
\label{sec: Auswertung}

\chapter{Literaturverzeichnis}

      % \begin{figure}[h]
      %       \centering
      %       \begin{minipage}[h]{0.49\linewidth}
      %       \centering
      %       \bilder{"./sources/himmel_und_wolken (www.umweltbundesamt.de)"}
      %       \caption{Betrachter:in auf der Erde}\cite{Himmel}
      %       \label{fig:himmel}
      %       \end{minipage}
      %       \hfill
      %       \begin{minipage}[h]{0.49\linewidth}
      %       \centering           
      %       \bilder{"./sources/erde (www.umweltbundesamt.de)"}
      %       \caption{Betrachter:in im Weltall auf die Erde}\cite{ErdeausWeltall}
      %       \label{fig:ErdeausWeltall}
      %       \end{minipage}
      % \end{figure}

      \printbibliography
\end{document}
