\documentclass[ngerman,BCOR=4mm]{tudscrreprt}

\usepackage[T1]{fontenc}
\usepackage[ngerman=ngerman]{hyphsubst}
\usepackage[ngerman]{babel}
\usepackage{isodate}
\usepackage[hidelinks]{hyperref}
\usepackage{amsmath, amsfonts, amssymb, amsthm} %% mathematics tools
\usepackage{csquotes}
\usepackage[draft]{listofsymbols}
\usepackage[backend=biber, style=numeric]{biblatex} %% Literature citing engine

%__________Definitions_____________________________

% Literaturverzeichnis
\addbibresource{Literatur.bib}


% Symbolverzeichnis
\renewcommand{\symheadingname}{Symbolverzeichnis}
\opensymdef
      \newsym[afd]{al}{\alpha}
\closesymdef


% wird für Symbolverzeichnis benutzt, sonst gibt es Fehler
\DeclareOldFontCommand{\bf}{\normalfont\bfseries}{\mathbf}


% Math
\theoremstyle{plain} % text is cursive
\newtheorem{theorem}{Theorem}
\newtheorem{lemma}[theorem]{Lemma}  %% [theorem] means same numbering for theorem and lemma
\newtheorem{proposition}[theorem]{Proposition}
\newtheorem{corollary}[theorem]{Korollar}

\theoremstyle{definition} % text is "upright"
\newtheorem{definition}[theorem]{Definition}
\newtheorem{example}[theorem]{Beispiel}

\theoremstyle{remark}
\newtheorem{remark}[theorem]{Bermerkung}

\begin{document}
\selectlanguage{ngerman}

%__________Deckblatt_Allg_Infos_____________________________
\faculty{Fakultät Mathematik}
\institute{Institut für Numerische Mathematik}
% \chair{Professur für Numerik partieller Differentialgleichungen}
\date{23.08.2024}
\title{%
      Gewichtete Eigenwert-Zählung auf einem Intervall
}
\subject{bachelor}
\graduation[B.Sc.]{Bachelor of Science}
\author{%
      Noah Göpel%
      \matriculationnumber{5029415}%
      \dateofbirth{01.02.2003}%
      \placeofbirth{Riesa}%
}
\matriculationyear{2021}
% \supervisor{}
\professor{Prof. Dr. Oliver Sander}
\maketitle

\clearpage
\tableofcontents
\clearpage
\listofsymbols
\clearpage

\chapter{Einleitung}
\label{sec: Einleitung}
      
      Diese Ausarbeitung beschäftigt sich mit numerischen Verfahren, um sicherzustellen, dass für ein gegebenes mechanisches System keine Eigenwerte in einem vorgegebenem festen Intervall liegen.
      % hier muss noch einiges getan werden
      Damit kann sichergestellt werden, dass die Eigenfrequenzen des Systems nicht so liegen, dass es zu einer Selbsterregung kommt.

      Um diese Anforderung zu erfüllen, werden die Eigenwerte des Matrix Pencils auf diesem Intervall gewichtet gezählt. Man erhält ein Minimierungsproblem,
      in welchem es gilt, einen Design-Paramerter so anzupassen, dass auf dem Intervall kein Eigenwert des entsprechenden Systems mehr vorhanden ist.
      Dazu werden in den Kapiteln \ref{sec: MS Matrizen} und \ref{sec: EW Problem} die theoretischen Grundlagen gelegt. Anschließend werden in Kapitel \ref{sec: Quellen} die entscheidende Identität von Futamura und wichtige Überlegungen ausgeführt.

      Daraufhin werden diese Überlegungen in Kapitel \ref{sec: Programmieren} anhand verschiedener Beispiele implementiert und getestet.

      Die Auswertung und Verbesserung der Implementation folgt in Kapitel \ref{sec: Verbesserungen}.

      % In dieser Ausarbeitung werden die gewichtete Zählung von Eigenwerten auf einem Intervall behandelt, um sicherzustellen, dass bei einem vorgegebenem mechanischem System keine Eigenwerte in einem bestimmten festen Intervall liegen.
      % Dazu wird in den ersten Kapiteln die nötige Theorie anhand der Quellen \cite{grundlageFutamura} und \cite{hauptteilTkachuk} beschrieben.
      % Hier werden Matrix Pencil und das verallgemeinerte Eigenwertproblem verwendet, um 
      % Hierzu nutzt man die Identität von Futamura, um die Zählung der Eigenwerte auf die zugrundeliegenden Matrizen zurückzuführen.


      % und anschließend in Kapitel \ref{sec: Programmieren} in Python angewendet.
      % Daraufhin werden in Kapitel \ref{sec: Verbesserungen} die Ergebnisse kritisch betrachtet und weitere Algorithmen eingeführt, um die Berechnungen zu beschleunigen.
      Zum Schluss werden in Kapitel \ref{sec: Auswertung} die Ergebnisse wiederholt und ein Ausblick in weiterführende Themen gegeben.

      Ferner werden in Kapitel \ref{sec: Quellen} eine Approximation der Matrix-Spur vorgestellt und ebenfalls implementiert.

\chapter{Masse- und Steifigkeits-Matrizen}
\label{sec: MS Matrizen}
      \section{Die Massematrix}

      \section{Die Steifigkeitsmatrix}

      \section{Zusammenhang Eigenwert und Eigenfrequenz}

      \section{Das Beispiel: lineares System von Massen und Federn}

\chapter{Das verallgemeinerte Eigenwert-Problem}
\label{sec: EW Problem}
      \section{Der Matrix Pencil}

      \section{Das verallgemeinerte Eigenwertproblem}

      \section{Die Schur-Zerlegung}

\chapter{Quellen-Ausarbeitungen}
\label{sec: Quellen}
      \section{Die Identität von Futamura}

      \section{Die Hutchinson-Approximation}

      \section{Die Berechnungen nach Tkachuk}

\chapter{Implementierung}
\label{sec: Programmieren}
      \section{Erklärungen}

      \section{Implementation}

      \section{Grafiken}

\chapter{Verbesserungen}
\label{sec: Verbesserungen}
      \section{Auswertung bisheriger Implementation}

      \section{mögliche Verbesserungen}

      \section{Implementation und Auswertung der verbesserten Programme}

\chapter{Auswertung}
\label{sec: Auswertung}

\chapter{Literaturverzeichnis}
      \printbibliography
\end{document}
